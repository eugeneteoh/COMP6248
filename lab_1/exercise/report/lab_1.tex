\documentclass[10pt, twocolumn]{article}

\makeatletter
\renewcommand*{\fps@figure}{!htb}
\renewcommand*{\fps@table}{!htb}
\makeatother

\usepackage{sectsty}
\sectionfont{\fontsize{11}{11}\selectfont}
\subsectionfont{\fontsize{10}{11}\selectfont}

\usepackage[compact]{titlesec}
\titlespacing{\section}{0pt}{2ex}{1ex}
\titlespacing{\subsection}{0pt}{1ex}{1ex}
\titlespacing{\subsubsection}{0pt}{0.5ex}{1ex}

\setlength{\parskip}{0cm}
\setlength{\parindent}{1em}

\usepackage{geometry}
 \geometry{
 a4paper,
 total={170mm,257mm},
 left=20mm,
 top=20mm,
 }
\usepackage[utf8]{inputenc}
\usepackage[hidelinks]{hyperref}
\usepackage{amsmath, bm}
\usepackage[ruled,vlined]{algorithm2e}
\usepackage{amssymb}
\usepackage{graphicx}
\usepackage{float}
\usepackage{booktabs}
\usepackage[parfill]{parskip}
\usepackage{comment}
\usepackage{subcaption}

\usepackage{listings}
\lstset{
    language=Python,
    breaklines=true,
    breakatwhitespace=true,
    basicstyle=\footnotesize,
    frame=lines
}
\usepackage[capitalise, nameinlink]{cleveref}

\usepackage[sorting=none, style=verbose]{biblatex}
\addbibresource{lab_1.bib}

\usepackage{titling}
\setlength{\droptitle}{-1cm}

\title{\Large COMP6248 Lab 1 Exercise -- Playing with gradients and matrices in PyTorch}
\author{\small Wei Chien Teoh (Eugene)\\\bigskip \href{mailto:wct1c16@soton.ac.uk}{wct1c16@soton.ac.uk}}
\date{\small 9 April 2021}

\begin{document}

\maketitle

\section*{Introduction}

The code implementation is stored in a Github repository \footcite{github}. The results are seeded using \lstinline{torch.manual_seed(0)} to provide reproducible results.

\section{Implement a matrix factorisation using gradient descent}

\subsection{Implement gradient-based factorisation}

\begin{lstlisting}
from typing import Tuple
import torch

def sgd_factorise(A: torch.Tensor, rank: int, num_epochs=1000, lr=0.01) -> Tuple[torch.Tensor, torch.Tensor]:
    m, n = A.shape
    U = torch.rand((m, rank))
    V = torch.rand((n, rank)) 

    for epoch in range(num_epochs):
        for r in range(m):
            for c in range(n):
                e = A[r, c] - (U[r] @ V[c].T)
                U[r] += lr * e * V[c]
                V[c] += lr * e * U[r]

    return U, V
\end{lstlisting}

\subsection{Factorise and compute reconstruction error} \label{sec:reconstruction}

\begin{align*}
    \begin{split}
    \hat{\pmb{U}} &= 
    \begin{bmatrix}
        0.6168 & -0.1530\\
        0.4108 &  1.5961\\
        1.0798 &  1.1800
    \end{bmatrix},\\
    \hat{\pmb{V}} &= 
    \begin{bmatrix}
        0.8126 &  1.8290\\
        0.7836 & -0.2088\\
        0.8384 & 1.0195
    \end{bmatrix},\\
    \text{Loss} &= 0.12197002023458481
    \end{split}
\end{align*}

\section{Compare your result to truncated SVD}

\subsection{Compare to the truncated-SVD}

\begin{align*}
    \begin{split}
    \pmb{U}_t &=
    \begin{bmatrix}
        -0.0801 & -0.7448 &  0.6625\\
        -0.7103 &  0.5090 &  0.4863\\
        -0.6994 & -0.4316 & -0.5697
    \end{bmatrix},\\
    \pmb{S}_t &=
    \begin{bmatrix}
        5.3339 & 0.6959 & 0.0000
    \end{bmatrix},\\
    \pmb{V}_t &=
    \begin{bmatrix}
        -0.8349 &  0.2548 &  0.4879\\
        -0.0851 & -0.9355 &  0.3430\\
        -0.5439 & -0.2448 & -0.8027
    \end{bmatrix},\\
    \text{Loss} &= 0.12191088497638702
    \end{split}
\end{align*}

The reconstruction loss of the truncated SVD is almost identical to the results in \cref{sec:reconstruction}. This is explained by the Eckart-Young-Mirsky theorem \footcite{eckart_approximation_1936}. The Eckart-Young-Mirsky theorem states that the singular value decomposition of a matrix $\pmb{D}$ can be approximated with subject to $rank(\pmb{D}) \leq r$. 

\section{Matrix completion}

\subsection{Implement masked factorisation}

\begin{lstlisting}
def sgd_factorise_masked(A: torch.Tensor, M: torch.Tensor, rank: int, num_epochs=1000, lr=0.01) -> Tuple[torch.Tensor, torch.Tensor]:
    m, n = A.shape
    U = torch.rand((m, rank))
    V = torch.rand((n, rank)) 

    for epoch in range(num_epochs):
        for r in range(m):
            for c in range(n):
                if M[r, c]:
                    e = A[r, c] - (U[r] @ V[c].T)
                    U[r] += lr * e * V[c]
                    V[c] += lr * e * U[r]

    return U, V
\end{lstlisting}

\subsection{Reconstruct a matrix}

\begin{align*}
    \begin{split}
    \hat{\pmb{U}} &=
    \begin{bmatrix}
        0.6501 & -0.1515\\
        0.2346 &  1.2954\\
        1.1472 & 1.2511
    \end{bmatrix},\\
    \hat{\pmb{V}} &=
    \begin{bmatrix}
        0.9019 &  1.5197\\
        0.8868 & -0.1223\\
        0.5406 &  1.3175
    \end{bmatrix},\\
    \hat{\pmb{A}} = \hat{\pmb{U}} \hat{\pmb{V}}^T &=
    \begin{bmatrix}
        0.3561 & 0.5951 & 0.1518\\
        2.1802 & 0.0496 & 1.8334\\
        2.9360 & 0.8643 & 2.2685
    \end{bmatrix},\\
    \text{Loss} &= 1.4483590126037598
    \end{split}
\end{align*}

Although the matrix completion approximation of $\hat{\pmb{A}}$ is not identical with $\hat{\pmb{A}}$, the gradient descent-based approach of minimization provides reasonable results in the recovery of the missing values subject to rank $r$.

% \printbibliography

\end{document}